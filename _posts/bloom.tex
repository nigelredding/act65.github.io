\documentclass[10pt]{amsart}
\usepackage{/home/nigel/Dropbox/mystyle}
\usepackage{graphicx}
\usepackage{listings,minted}
\graphicspath{{./}}

\title{Notes on Bloom and Cuckoo Filters}
\author{Nigel Redding}
\date{\today}

\begin{document}

\maketitle

\tableofcontents

Let $S=\{s_1,\ldots,s_m\}$ 
be a set of elements belonging to some larger set $U$. Bloom and Cuckoo filters
are data structures which allow us to check, with some fixed probability of
getting a false positive, whether an element $x \in U$ is a member of $S$.
This is done in constant time for both Bloom and Cuckoo filters.

\section{Bloom Filters}
Let $U$ be a set and let $S = \{s_1,\ldots,s_m\}$ be a finite subset.
Let $A$ be an array of $n$ bits, with $A[0],\ldots,A[n-1]$ initially
being $0$. Let $h_1,\ldots,h_k : U \to \{0,1,\ldots,n-1\}$ be $k$
hash functions. We make the assumtion that the functions map
each $x \in U$ to a random number uniformly in $\{0,1,\ldots,n\}$,
and that $h_1,\ldots,h_n$ are independent.\\

To insert an element $x$ into $A$, we set
$$A[h_1(x)] := 1, \ldots, A[h_k(x)] := 1.$$
In our Go code, this is implemented by the
\texttt{insert} function. Typically, we begin by
inserting all $s \in S$ into $A$. After this,
there is the operation \texttt{check}, which
tells us whether an element $x \in U$ (with some chance
of a false positive) is a member of $S$. Both of these operations
are clearly constant time operations.\\

For given $m,n$ and $k$, we can compute the probability that
\texttt{check} returns \texttt{true} for a given $x \in U \setminus S$.
Indeed, for a given bit, the probability that that bit is still $0$
after inserting all of $S$ into $A$ is $$\left ( 1 - \dfrac{1}{n} \right )^{km} .$$
Hence, the probability that a given bit is $1$ after insertion is 
$$1 - \left ( 1 - \dfrac{1}{n} \right )^{km}.$$ Given that the hashes
are $k$ bits long, the probability of getting a false positive
is $$\left ( 1 - \left ( 1 - \dfrac{1}{n} \right )^{km} \right )^k.$$
Given that $\lim_n (1 + \frac{x}{n})^n = e^x$, we see that
$$\left ( 1 - \left ( 1 - \dfrac{1}{n} \right )^{km} \right )^k
\approx (1 - e^{-km/n})^k.$$ Note that we made heavy use
of the independece of $h_1,\ldots,h_k$. We set
$g(k) := (1 - e^{-km/n})^k$. Using calculus, we see that
$g$ has a global minimum at $k = \log(2) \cdot \frac{n}{m}$. At
this specific $k$, the probability of getting a false positive
is $$\left (\dfrac{1}{2} \right )^k \approx (0.6185)^{n/m}.$$\\

See \href{http://0pointer.net/blog/projects/bloom.html}{here} for a short discussion
on how to chose your $k$ hash functions.

\section{Cuckoo Filters}

Bloom filters provide $O(1)$ \texttt{Insert} and $O(1)$ \texttt{Lookup}, but they
do not support deletion. We describe another data structure, which provides
(amortized) $O(1)$ \texttt{Insert}, \texttt{Lookup} and \texttt{Delete}. We
follow \href{https://www.cs.cmu.edu/~dga/papers/cuckoo-conext2014.pdf}{this paper}.
A Cuckoo Filter $F$ consists of an array of $m$ buckets, where each bucket holds
$b$ \emph{fingerprints} (explained later) of length $f$ bits. We insert $n$ items into $F$.
$F$ is equipped with a hash function $h$ which maps $U$ into $\{0,1,\ldots,m\}$.\\

We go trough the implementation of the operations \texttt{Insert,Lookup} and \texttt{Delete}.\\

For each item $x \in U$, there is a hash $h(x)$ (let's say it's a 32 bit integer)
and a fingerprint, which is assigned to be the lowest $f$ bytes of $h(x)$ \footnote{In my
implementation, we take $f=8$ (constant), which makes it impossible to implement a good
Cuckoo Filter in many cases. This will be fixed later.} To insert $x$ into $F$, we insert
its fingerprint $f(x)$. We do this as follows.\\

\texttt{Insert(x)} returns \texttt{true} if $x$ was inserted correctly, and \texttt{false}
otherwise.
\begin{minted}{go}
Insert(x) {
	index1 := h(x)
	fprint := first p bits of h(x)
	index2 := index1 ^ h(fprint)
	
	if buckets[index1] not full {
		add fprint to buckets[index1]
		return true
	}
	if buckets[index2] not full {
		add fprint to buckets[index2]
		return true
	}
	
	i := chose index1 or index2 randomly
	for n := 0; n < maxKicks; n++ {
		swap fprint with a random entry in buckets[i]
		fph := h(fprint)
		index = index ^ fph
		if buckets[index] not full {
			add fprint to buckets[index]
			return true
		}
	}
}
\end{minted}

Let's step through this line by line. 

\begin{enumerate}
\item When we insert $x$, there are two locations in can be
placed in, \texttt{index1 = h(x)} or its alternate location 
\texttt{index2 = index1 $^\wedge$ h(fprint)}. We can recover \texttt{index1} from 
\texttt{index2} by \texttt{index1 = index2 $^\wedge$ \textasciitilde h(fprint)}.

\item Following this, we attempt to add into the bucket at index1 and index2. If either
succeed, we return true.

\item If neither suceeed, we pick index1 or index2 at random, and assign it to
i. Then, for a certain number of steps (\texttt{maxKicks = 500} in our reference)
we do the following.

\item Swap our fingerprint fprint with a random entry in bucket[i]. Find
its alternate location, and try to put it in there. If you fail, repeat with another
fingerprint in buckets[i]. Repeat this step until we either return \texttt{true}
or we go through \texttt{maxKicks} steps.
\end{enumerate}

The \texttt{Lookup} and \texttt{Delete} operations are short and easy to understand, so
I refer the reader to page 5 of \href{https://www.cs.cmu.edu/~dga/papers/cuckoo-conext2014.pdf}{this paper}.
\end{document}

